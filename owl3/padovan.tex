\documentclass[11pt]{article}
\usepackage[utf8]{inputenc}
\usepackage{amsfonts}
\usepackage[fleqn]{amsmath}
\usepackage{mathtools}
\usepackage[dvipsnames]{xcolor}
\usepackage[left=35mm,top=26mm,right=26mm,bottom=15mm]{geometry}

\usepackage{caption}
\usepackage{subcaption}

\def \nonneg {\mathbb{Z}_{\leq 0}}
\def \lf {\newline}

\def \subsorange [#1]{\mathbf{\textcolor{orange}{#1}}}
\setlength{\parindent}{0pt}

\title{Owl 3: Padovan Numbers}
\author{Cole Gannon}
\date{February 28th, 2023 \textcolor{red}{(late)}}

\begin{document}
\maketitle
The Padovan numbers are defined by a function ${P(n \in \nonneg)}$\ where\lf

$P(0)=P(1)=P(2)=1$\lf

$\forall n \geq 3:P(n)=P(n-2)+P(n-3)$\lf

To make this easier, let's just always assume that $n\in\nonneg$ so I don't have to repeat myself.
Let's also work out the first few Padovan numbers for ease of simplification later.  \lf

\begin{tabular}{||c c c c c c c c c||}
   \hline n   &\vline\ \ 0 & 1 & 2 & 3 & 4 & 5 & 6 & 7 \\
   \hline P(n)&\vline\ \ 1 & 1 & 1 & 2 & 2 & 3 & 4 & 5 \\
   \hline
  \end{tabular}

\section*{Claim 1: $\displaystyle\sum_{m=0}^{n} P(m) = P(n+5)-2$}

\subsection*{Proof by induction}

First, let us prove our three base cases: $n=0,1,2$.
In both claims, these will be our bases cases. I'm not really sure why but it felt like it was important to cover these since $P(n)=P(n-2)+P(n-3) \iff n>=3$. I do not have a better explanation :/

\begin{table}[h]
   \begin{subtable}[h]{0.45\textwidth}
      \subsubsection*{Base case $n=\subsorange[0]$}
      \begin{alignat*}{2}
         \sum_{m=0}^{\subsorange[0]}P(m) \\
         &=P(0) &\phantom{+P(1)} \\
         &=1 \\
         \\ \hline \\
         P(\subsorange[0]+5)-2 \\
         &=3-2 \\
         &=1 \\
         &\phantom{=1}
      \end{alignat*}
      Both sides are equivalent. \textbf{QED} $n=\subsorange[0]$
   \end{subtable}
   \hfill
   \begin{subtable}[h]{0.45\textwidth}
      \subsection*{Base case $n=\subsorange[1]$}
      \begin{alignat*}{2}
         \sum_{m=0}^{\subsorange[1]}P(m) \\
         &=P(0)+P(1) \\
         &=2 \\
         \\ \hline \\
         P(\subsorange[1]+5)-2 \\
         &=P(6)-2 \\
         &=4-2 \\
         &=2
      \end{alignat*}
      Both sides are equivalent. \textbf{QED} $n=\subsorange[1]$
   \end{subtable}
\end{table}

\pagebreak

\subsection*{Base case $n=\subsorange[2]$}
\begin{table}[h]
   \begin{subtable}[t]{0.45\textwidth}
      \begin{alignat*}{2}
         \sum_{m=0}^{\subsorange[1]}P(m) \\
         &=P(0)+P(1) \\
         &=2
      \end{alignat*}
   \end{subtable}
   \vline
   \begin{subtable}[t]{0.45\textwidth}
      \begin{alignat*}{2}
         P(\subsorange[1]+5)-2 \\
         &=P(6)-2 \\
         &=4-2 \\
         &=2
      \end{alignat*}
   \end{subtable}
\end{table}

Both sides are equivalent. \textbf{QED} $n=\subsorange[2]$; \textbf{QED} base cases!

\subsection*{Inductive hypothesis}

Assume that the claim is true for $n$. It is sufficient to prove that the claim is true for $n+1$. In other words:\lf

Given
\[
   \sum_{m=0}^{n} P(m) = P(n+5)-2
\]

Show
\[
   \sum_{m=0}^{n + 1} P(m) = P((n+1)+5)-2
\]

\begin{table}[h]
   \begin{subtable}[t]{0.45\textwidth}
      \begin{alignat*}{2}
         &\left[\sum_{m=0}^{n+1} P(m)\right] \\
         &=\textcolor{purple}{\left[\sum_{m=0}^{n} P(m)\right]}+P(n+1) \\
         &=\textcolor{purple}{\left[\textcolor{blue}{P(n+5)}-2\right]\text{\tiny(by assumption)}}+P(n+1) \\
         &=\textcolor{blue}{P(n+3)+P(n+2)}+P(n+1)-2
      \end{alignat*}
   \end{subtable}
   \vline
   \begin{subtable}[t]{0.45\textwidth}
      \begin{alignat*}{2}
         &\textcolor{ForestGreen}{P(n+6)}-2 \\
         &=\textcolor{ForestGreen}{\left[\textcolor{RedOrange}{P(n+4)}+P(n+3)\right]}-2 \\
         &=\textcolor{RedOrange}{P(n+2)+P(n+1)}+P(n+3)-2 \\
         &=P(n+3)+P(n+2)+P(n+1)-2
      \end{alignat*}
   \end{subtable}
\end{table}

Symbolic equivalence achieved:
\begin{align*}
   \sum_{m=0}^{n + 1} P(m) &= P((n+1)+5)-2 \\
   \rotatebox{40}{$\,\leftarrow$}\phantom{aaaaaaaaa} &\phantom{aaaaaaaaaaaa} \rotatebox{90}{$\,\leftarrow$}\\
   P(n+3)+P(n+2)+P(n+1)-2 &= P(n+3)+P(n+2)+P(n+1)-2
\end{align*}

\textbf{QED Claim 1!}

\pagebreak

\section*{Claim 2: $\displaystyle\sum_{m=0}^{n} P(2m) = P(2n+3)-1$}

\subsection*{Proof by induction}

Once again, let us prove our three base cases:

\begin{table}[h]
   \begin{subtable}[t]{0.45\textwidth}
      \subsubsection*{Base case $n=\subsorange[0]$}
      \begin{alignat*}{2}
         \sum_{m=0}^{\subsorange[0]}P(2m)& \\
            &=P(0) \\
            &=\mathbf{1} \\
            &\phantom{=2} \\
         P(2(\subsorange[0])+3)-1& \\
            &=P(3)-1 \\
            &=2-1 \\
            &=\mathbf{1} \\
      \end{alignat*}
      \textbf{QED} case $n=\subsorange[0]$
   \end{subtable}
   \hfill
   \vline
   \hfill
   \begin{subtable}[t]{0.45\textwidth}
      \subsubsection*{Base case $n=\subsorange[1]$}
      \begin{alignat*}{2}
         \sum_{m=0}^{\subsorange[1]}P(2m)& \\
            &=P(0)+P(2) \\
            &=1+1 \\
            &=\mathbf{2} \\
         P(2(\subsorange[1])+3)-1& \\
            &=P(5)-1 \\
            &=3-1 \\
            &=\mathbf{2} \\
      \end{alignat*}
      \textbf{QED} case $n=\subsorange[1]$
   \end{subtable}
\end{table}

\subsubsection*{Base case $n=\subsorange[2]$}
\begin{table}[h]
   \begin{subtable}[t]{0.45\textwidth}
      \begin{alignat*}{2}
         \sum_{m=0}^{\subsorange[2]}P(2m)& \\
            &=P(0)+P(2)+P(4) \\
            &=1+1+2 \\
            &=\mathbf{4} \\
      \end{alignat*}
   \end{subtable}
   \hfill
   \vline
   \hfill
   \begin{subtable}[t]{0.45\textwidth}
      \begin{alignat*}{2}
         P(2(\subsorange[2])+3)-1& \\
            \\
            &=P(7)-1 \\
            &=5-1 \\
            &=\mathbf{4} \\
      \end{alignat*}
   \end{subtable}
\end{table}
\textbf{QED} case $n=\subsorange[2]$; \textbf{QED} base cases!

\subsection*{Inductive hypothesis}
Assume that the claim is true for $n$. It is sufficient to prove that the claim is true for $n+1$. In other words:\lf

Given
\[
   \sum_{m=0}^{n} P(2m) = P(2n+3)-1
\]

Show
\[
   \sum_{m=0}^{n + 1} P(2m) = P(2(n+1)+3)-1
\]

\pagebreak

\begin{table}[h]
   \begin{subtable}[t]{0.45\textwidth}
      \begin{alignat*}{2}
         &\left[\sum_{m=0}^{n+1} P(2m)\right] \\
         &=\textcolor{purple}{\left[\sum_{m=0}^{n} P(2m)\right]}+P(2(n+1)) \\
         &=\textcolor{purple}{\left[P(2n+3)-1\right]\text{\tiny(by assumption)}}+P(2n+2) \\
         &=P(2n+3)+P(2n+2)-1
      \end{alignat*}
   \end{subtable}
   \vline
   \begin{subtable}[t]{0.45\textwidth}
      \begin{alignat*}{2}
         &P(\textcolor{ForestGreen}{2(n+1)+3})-1 \\
         &=\textcolor{RedOrange}{P(\textcolor{ForestGreen}{2n+5})}-1 \\
         &=\textcolor{RedOrange}{\left[\textcolor{black}{P(2n+5-2)+P(2n+5-3)}\right]}-1 \\
         &=P(2n+3)+P(2n+2)-1
      \end{alignat*}
   \end{subtable}
\end{table}

Symbolic equivalence achieved:
\begin{align*}
   \sum_{m=0}^{n + 1} P(2m) &= P(2(n+1)+3)-1 \\
   \rotatebox{50}{$\,\leftarrow$}\phantom{aaaaaaaaa} &\phantom{aaaaaaaaaaaa} \rotatebox{90}{$\,\leftarrow$} \\
   P(2n+3)+P(2n+2)-1 &= P(2n+3)+P(2n+2)-1
\end{align*}

\textbf{QED Claim 2!}

You know, the hardest part of this was using \LaTeX.

\end{document}
