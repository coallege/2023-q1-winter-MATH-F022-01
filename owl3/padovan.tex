\documentclass[11pt]{article}
\usepackage[utf8]{inputenc}
\usepackage{amsfonts}
\usepackage{mathtools}
\usepackage{xcolor}
\usepackage[left=35mm,top=26mm,right=26mm,bottom=15mm]{geometry}

\def \nonneg {\mathbb{Z}_{\leq 0}}
\def \lf {\newline}

\def \subsorange [#1]{\mathbf{\textcolor{orange}{#1}}}
\setlength{\parindent}{0pt}

\title{Owl 3: Padovan Numbers}
\author{Cole Gannon}
\date{February 28th, 2023 \textcolor{red}{(late)}}

\begin{document}
\maketitle
The Padovan numbers are defined by a function ${P(n \in \nonneg)}$\ where\lf

$P(0)=P(1)=P(2)=1$\lf

$\forall n \geq 3:P(n)=P(n-2)+P(n-3)$\lf

To make this easier, let's just always assume that $n\in\nonneg$ so I don't have to repeat myself.
Let's also work out the first few Padovan numbers for ease of simplification later.\lf


\begin{tabular}{||c c||}
   \hline n & P(n) \\ [0.5ex] \hline
   \hline 0 & 1 \\
   \hline 1 & 1 \\
   \hline 2 & 1 \\
   \hline 3 & 2 \\
   \hline 4 & 2 \\
   \hline 5 & 3 \\
   \hline 6 & 4 \\
   \hline
  \end{tabular}

\section*{Claim 1: $\displaystyle\sum_{m=0}^{n} P(m) = P(n+5)-2$}

\subsection*{Proof by induction}

First, let us prove our three base cases: $n=0,1,2$

\subsubsection*{Base case $n=\subsorange[0]$}

Left Hand Side:
\[
   \left[\sum_{m=0}^{\subsorange[0]}P(m)\right]=P(0)=1
\]

Right Hand Side:
\begin{alignat*}{2}
   P(\subsorange[0]+5)-2 \\
   &=-2+P(5-2)         &&+\ P(5-3) \\
   &=-2+P(3)           &&+\ P(2) \\
   &=-2+P(3-1)+P(3-2)\ &&+\ P(2) \\
   &=-2+P(2)+P(1)      &&+\ P(2) \\
   &=-2+1+1+1 \\
   &=1
\end{alignat*}

\textbf{QED} $n=\subsorange[0]$

\pagebreak
\subsection*{Base case $n=\subsorange[1]$}

Left Hand Side:
\[
   \left[\sum_{m=0}^{\subsorange[1]}P(m)\right]=P(0)+P(1)=2
\]

Right Hand Side:
\begin{alignat*}{2}
   P(\subsorange[1]+5)-2 \\
   &=-2+P(6) \\
   &=-2+P(6-2) + P(6-3) \\
   &=-2+P(4) + P(3) \\
\end{alignat*}

\end{document}
