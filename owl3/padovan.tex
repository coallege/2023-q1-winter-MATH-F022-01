\documentclass[11pt]{article}
\usepackage[utf8]{inputenc}
\usepackage{amsfonts}
\usepackage[fleqn]{amsmath}
\usepackage{mathtools}
\usepackage[dvipsnames]{xcolor}
\usepackage[left=35mm,top=26mm,right=26mm,bottom=15mm]{geometry}

\usepackage{caption}
\usepackage{subcaption}

\def \nonneg {\mathbb{Z}_{\leq 0}}
\def \lf {\newline}

\def \subsorange [#1]{\mathbf{\textcolor{orange}{#1}}}
\setlength{\parindent}{0pt}

\title{Owl 3: Padovan Numbers}
\author{Cole Gannon}
\date{February 28th, 2023 \textcolor{red}{(late)}}

\begin{document}
\maketitle
The Padovan numbers are defined by a function ${P(n \in \nonneg)}$\ where\lf

$P(0)=P(1)=P(2)=1$\lf

$\forall n \geq 3:P(n)=P(n-2)+P(n-3)$\lf

To make this easier, let's just always assume that $n\in\nonneg$ so I don't have to repeat myself.
Let's also work out the first few Padovan numbers for ease of simplification later.  \lf

\begin{tabular}{||c c c c c c c c||}
   \hline n   &\vline\ \ 0 & 1 & 2 & 3 & 4 & 5 & 6 \\
   \hline P(n)&\vline\ \ 1 & 1 & 1 & 2 & 2 & 3 & 4 \\
   \hline
  \end{tabular}

\section*{Claim 1: $\displaystyle\sum_{m=0}^{n} P(m) = P(n+5)-2$}

\subsection*{Proof by induction}

First, let us prove our three base cases: $n=0,1,2$

\begin{table}[h]
   \begin{subtable}[h]{0.45\textwidth}
      \subsubsection*{Base case $n=\subsorange[0]$}
      \begin{alignat*}{2}
         \sum_{m=0}^{\subsorange[0]}P(m) \\
         &=P(0) &\phantom{+P(1)} \\
         &=1 \\
         \\ \hline \\
         P(\subsorange[0]+5)-2 \\
         &=3-2 \\
         &=1 \\
         &\phantom{=1}
      \end{alignat*}
      Both sides are equivalent. \textbf{QED} $n=\subsorange[0]$
   \end{subtable}
   \hfill
   \begin{subtable}[h]{0.45\textwidth}
      \subsection*{Base case $n=\subsorange[1]$}
      \begin{alignat*}{2}
         \sum_{m=0}^{\subsorange[1]}P(m) \\
         &=P(0)+P(1) \\
         &=2 \\
         \\ \hline \\
         P(\subsorange[1]+5)-2 \\
         &=P(6)-2 \\
         &=4-2 \\
         &=2
      \end{alignat*}
      Both sides are equivalent. \textbf{QED} $n=\subsorange[1]$
    \end{subtable}
\end{table}

\subsection*{Base case $n=\subsorange[2]$}
\begin{alignat*}{2}
   \sum_{m=0}^{\subsorange[1]}P(m) \\
   &=P(0)+P(1) \\
   &=2 \\
   \\ \hline \\
   P(\subsorange[1]+5)-2 \\
   &=P(6)-2 \\
   &=4-2 \\
   &=2
\end{alignat*}
Both sides are equivalent. \textbf{QED} $n=\subsorange[2]$; \textbf{QED} Base cases!

\subsection*{Inductive hypothesis}

Assume that the claim is true for $n$. It is sufficient to prove that the claim is true for $n+1$. In other words:\lf

Given
\[
   \sum_{m=0}^{n} P(m) = P(n+5)-2
\]

Show
\[
   \sum_{m=0}^{n + 1} P(m) = P((n+1)+5)-2
\]

\begin{table}[h]
   \begin{subtable}[t]{0.45\textwidth}
      \begin{alignat*}{2}
         &\left[\sum_{m=0}^{n+1} P(m)\right] \\
         &=\textcolor{purple}{\left[\sum_{m=0}^{n} P(m)\right]}+P(n+1) \\
         &=\textcolor{purple}{\left[\textcolor{blue}{P(n+5)}-2\right]\text{\tiny(by assumption)}}+P(n+1) \\
         &=\textcolor{blue}{P(n+3)+P(n+2)}+P(n+1)-2
      \end{alignat*}
   \end{subtable}
   \vline
   \begin{subtable}[t]{0.45\textwidth}
      \begin{alignat*}{2}
         &\textcolor{ForestGreen}{P(n+6)}-2 \\
         &=\textcolor{ForestGreen}{\left[\textcolor{violet}{P(n+4)}+P(n+3)\right]}-2 \\
         &=\textcolor{violet}{P(n+2)+P(n+1)}+P(n+3)-2 \\
         &=P(n+3)+P(n+2)+P(n+1)-2
      \end{alignat*}
   \end{subtable}
\end{table}

\[
   P(n+3)+P(n+2)+P(n+1)-2 = P(n+3)+P(n+2)+P(n+1)-2
\]

$\displaystyle\sum_{m=0}^{n + 1} P(m)$ is symbolically equivalent to $P((n+1)+5)-2$. \textbf{QED Claim 1!}

\end{document}
